\documentclass{beamer}

\usetheme{Madrid}
\setbeamersize{text margin right=0.8cm}
\setbeamersize{text margin left=0.8cm}

\usepackage[utf8]{inputenc}
\usepackage[T1]{fontenc}
\usepackage[brazil]{babel}
\usepackage{ragged2e}
\usepackage{amsmath}
\usepackage{amssymb}
\usepackage{graphicx}

\theoremstyle{definition}
\newtheorem{thm}{Teorema}[section]

\title{Inferência Bayesiana}
\author{Maicon C. Germano}
\institute{IBB UNESP}
\date{\today}

\begin{document}

\begin{frame}
  \titlepage
\end{frame}

\begin{frame}{Sumário}
  \tableofcontents
\end{frame}

\section{Introdução}

\subsection{O que é inferência Bayesiana?}
\begin{frame}{O que é inferência Bayesiana?}
\begin{itemize}
\item \justifying{A inferência Bayesiana é um método estatístico que leva em conta informações prévias sobre o que está sendo analisado.}
\vspace{0.2cm}
\item Ela combina essas informações com os novos dados que estão sendo analisados para chegar a uma conclusão mais precisa e confiável.
\vspace{0.2cm}
\item O parâmetro populacional $\theta$ é tratado como uma quantidade aleatória, o que leva a abordagens substancialmente diferentes à modelagem e inferência em comparação à inferência clássica.
\end{itemize}
\end{frame}

\begin{frame}{O que é inferência Bayesiana?}
\begin{itemize}
\item \justifying{A inferência Bayesiana é baseada em $f(\theta \mid y)$ ao invés de se basear em $f(y \mid \theta)$, ou seja, na probabilidade do parâmetro condicional aos dados obtidos.}
\vspace{0.2cm}
\item Para utilizar a inferência Bayesiana, é necessário especificar uma distribuição a priori de probabilidades $f(\theta)$ que representa as crenças sobre a distribuição de $\theta$ antes de se considerar qualquer informação proveniente dos dados.
\vspace{0.2cm}
\item A noção de distribuição a priori para o parâmetro $\theta$ está no cerne do pensamento bayesiano.

\end{itemize}
    
\end{frame}

\subsection{Inferência Bayesiana vs. Clássica}
\begin{frame}{Inferência Bayesiana vs. Clássica}
    
    \begin{itemize}
    
    \item \justifying{Existem diferentes maneiras de analisar dados e chegar a conclusões estatísticas. Duas dessas maneiras são a inferência Bayesiana e a inferência clássica.}
    \vspace{1cm}
    \item A inferência Bayesiana é um método que permite que se leve em conta informações prévias sobre o que está sendo analisado, como experiências anteriores ou conhecimentos sobre o assunto. Ela combina essas informações com os novos dados que estão sendo analisados, para chegar a uma conclusão mais precisa e confiável. 
    \end{itemize}
    
\end{frame}

\begin{frame}{Inferência Bayesiana vs. Clássica}
 \begin{itemize}
    \item \justifying{Já a inferência clássica é uma maneira mais tradicional de analisar dados, sem levar em conta informações prévias. Nesse método, são usados estimadores de máxima verossimilhança, que buscam encontrar a estimativa mais provável dos parâmetros do modelo que está sendo analisado.}
    \vspace{1cm}
    \item A principal diferença entre esses dois métodos é que a inferência Bayesiana leva em conta informações prévias, enquanto a inferência clássica não o faz. Isso pode fazer com que a inferência Bayesiana seja mais precisa e confiável em algumas situações.
    \end{itemize}
    
\end{frame}

\subsection{Distribuição a priori}
\begin{frame}{Distribuição a priori}
\begin{itemize}
\item \justifying{Ao tentar estimar um parâmetro $\theta$, é comum ter algum conhecimento ou crença prévia sobre o valor de $\theta$ antes de considerar os dados.
\vspace{0.2cm}
\item A distribuição a priori representa a crença prévia sobre o valor de $\theta$, antes de observar os dados.}
\vspace{0.2cm}
\item A informação da distribuição a priori é combinada com a informação dos dados para obter a distribuição a posteriori, que reflete a crença atualizada sobre o valor de $\theta$.
\end{itemize}
\end{frame}
\begin{frame}{Distribuição a Priori}
\begin{itemize}
\item \justifying{A distribuição a priori pode ser baseada em conhecimento prévio, experiências anteriores, especialistas ou modelos teóricos.}
\vspace{0.2cm}
\item \justifying{A distribuição a priori pode influenciar as inferências feitas a partir dos dados, pois pode levar a diferentes distribuições a posteriori e conclusões diferentes.}
\vspace{0.2cm}
\item A escolha da distribuição a priori pode ser subjetiva e pode levar a diferentes inferências entre diferentes indivíduos ou grupos.
\end{itemize}
    
\end{frame}

\section{Teorema de Bayes}
\subsection{Breve revisão}
\begin{frame}{Teorema de Bayes: revisão}
Em sua forma básica, o Teorema de Bayes é simplesmente um resultado de probabilidade
condicional:\\
\

\textbf{Teorema de Bayes}
\begin{thm}[1]
    Se A e B são dois eventos com $P(A)>0$ então:
    $$P(B|A)=\frac{P(A|B)P(B)}{P(A)}$$
\end{thm}
\end{frame}

\begin{frame}{Exemplo 1}

\justifying{\textbf{Exemplo 1}. Um procedimento de testes de diagnóstico para HIV é aplicado a uma população de alto risco; acredita-se que 10\% desta população é positiva para o HIV. O teste de diagnóstico é positivo para 90\% das pessoas que de fato são HIV-positivas, e negativo para 85\% das pessoas que não são HIV-positivas. Qual a probabilidade de resultados falso-positivo e falso-negativo?} \\

\textbf{Notação:}

$A$: a pessoa é HIV-positiva,
$B$: o resultado do teste é positivo.
Temos dados que $P(A)=0,1,P(B\mid A)=0,9$ e $P(\bar{B}\mid\bar{A})=0,85$. Então:

$$P(\text{falso positivo})=P(\bar{A} \mid B)=\frac{P(B \mid \bar{A}) P(\bar{A})}{P(B)}$$ 
$$=\frac{0,15 \times 0,9}{(0,15 \times 0,9)+(0,9 \times 0,1)}=0,6$$

\end{frame}

\begin{frame}{Continuação do exemplo 1}
De forma similar,
$$
\begin{aligned}
P(\text {falso negativo}) & =P(A \mid \bar{B}) \\ \\
& =\frac{P(\bar{B} \mid A) P(A)}{P(\bar{B})} \\ \\
& =\frac{0,1 \times 0,1}{(0,1 \times 0,1)+(0,85 \times 0,9)}=0,0129
\end{aligned}
$$
\end{frame}

\begin{frame}{Resumindo informação a posteriori}
\subsection{Intervalos de credibilidade}
\textbf{Intervalos de credibilidade}


\subsection{Teste de hipóteses}
    
\end{frame}

\section{Aplicação na radioterapia}

\begin{frame}{Aplicação na Radioterapia}
    
\end{frame}
 
\end{document}
